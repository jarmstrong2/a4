% Title:   UofT Art & Sciences Assignment Sample File
% Version: 1.00
% Author:  Kaveh Ghasemloo
% Date:    Sept. 28, 2012
%
% Licence: 
% This work is licensed under the Creative Commons Attribution-ShareAlike 3.0 Unported License. To view a copy of this license, visit http://creativecommons.org/licenses/by-sa/3.0/ or send a letter to Creative Commons, 444 Castro Street, Suite 900, Mountain View, California, 94041, USA.

\documentclass[10pt]{csc_assignment}
\usepackage[]{algorithm2e}
\usepackage{amsmath}
\usepackage{algpseudocode}
\usepackage{qtree}
\usepackage{tkz-graph}

% ----------------------------------------------------------------
% TODO: Enter the assignment number, your name, and your student number below
% ----------------------------------------------------------------
\AssignmentName{4}
\QuestionCount{4}
\StudentName{John Armstrong, Henry Ku}
\StudentNumber{993114492\textbackslash g2jarmst, 998551348\textbackslash g2kuhenr}

% ----------------------------------------------------------------
\begin{document}


\Acknowledgements{
% ----------------------------------------------------------------
% TODO: Write your acknowledgements below.
% ----------------------------------------------------------------

"We declare that we have not used any outside help in completing this assignment."

% ----------------------------------------------------------------
% Aacknowledgements ends
% ----------------------------------------------------------------
}
\begin{description}

\newpage
\item[Q1. The Mute Prison]
% ----------------------------------------------------------------
% TODO: Write your answer to the question below. 
% ----------------------------------------------------------------
~\\
\textbf{Claim:} The mute prison problem is NP-complete.\\
\textbf{Proof:}\\
\underline{1.} Show the mute prison problem is NP.\\
\underline{2.} Show the mute prison problem is NP-hard.\\

\underline{1.} Suppose we are given a certificate S and have access to value k and matrix T. We can verify that the certificate is satisfiable in the following way. Suppose each element in S represents an inmate. Verification would involve iterating on each inmate in the following way:\\
\begin{algorithm}[H]
 \LinesNumbered 
\For {inmate in S}{
j = 1\;
\While{j $\leqslant$ m}{
\If{T[inmate, j]}{
\For {(otherinmate $\neq$ inmate) in S}{
\If{T[otherinmate, j]}{
	S is not a subset of inmates who do don't speak the same language\;
}
}
}
j++\;
}
}
\end{algorithm}
Clearly, the verification that S is a subset where no two inmates speak the same language can run in polynomial time O(mn$^{2}$). Once this verification if complete all that is left to do is to verify that $\mid$S$\mid$ $\geqslant$ k, which is O(1). Therefore the mute prison problem is NP. $\blacksquare$\\

\underline{2.} To show that the mute prison problem is NP-hard we must perform a reduction using an NP-complete problem. We will use a reduction on NP-complete 3-SAT in CNF, in order to show 3-SAT $\leqslant_{p}$ Mute Prison Problem.\\

\textbf{Properties of Reduction}\\
Suppose that $\phi$ is an instance of 3-SAT and C$_{1}$, C$_{2}$, ..., C$_{m}$ are the clauses of $\phi$. By construction of 3-SAT in CNF we have C$_{i}$ = ($z_{i1}$ $\lor$ $z_{i2}$ $\lor$ $z_{i3}$). In the reduction each C$_{i}$'s boolean value will represent a boolean value for each language, L$_{i}$, spoken by some inmate(s), precisely, L$_{i}$ = C$_{i}$ = ($z_{i1}$ $\lor$ $z_{i2}$ $\lor$ $z_{i3}$). Each boolean value for L$_{i}$ has a specific mean:
\[
L_{i} = 
\begin{cases} 
      \hfill 1 \hfill & \text{if L$_{i}$ is spoken by at most 1 inmate} \\
      \hfill 0 \hfill & \text{if L$_{i}$ is spoken by at least 1 inmate} \\
  \end{cases}
\]\\
Producing L$_{1}$, L$_{2}$, ..., L$_{m}$ will take polynomial time since we iterate through each C$_{i}$ and perform a boolean or operation on each z$_{i}$ in C$_{i}$ which takes O(m).\\

Finally, the mute prison problem requires a matrix T to produce the subset of inmates S. Let T be an m x m matrix, so that no inmates are left without a language. The rows in T will represent inmates and the columns will represent languages such that column i represents L$_{i}$. The algortihm that peforms the reduction will iterate through each L$_{i}$. If L$_{i}$ = 1 then set T[i, i] = 1, else if L$_{i}$ =  0 then \mbox{T[1, i] = T[2, i] = ... = T[m, i] = 1}. Assigning all inmates to speak L$_{i}$, when L$_{i}$ = 0, will guarantee that $\mid$S$\mid$ = 0. Alternatively, $\forall$ i, if L$_{i}$ = 1 then $\mid$S$\mid$ = m. So that if $\phi$ is satisfies 3-SAT, then T will satisfy the mute prison problem if we set k = m. Again this process is polynomial as it iterates through m L$_{i}$'s and assigns at most m inmates the language L$_{i}$, so it will run O(m$^{2}$).\\
\textbf{$\phi$ of 3-SAT is satisfiable $\rightarrow$ L and k of mute prison problem is satisfiable}\\
Suppose $\phi$ of 3-SAT is satisfiable, then each clause C$_{1}$, C$_{2}$, ..., C$_{m}$ is satisfied. A set of L$_{1}$, ..., L$_{m}$ is produced such that $\forall$ L$_{i}$, L$_{i}$ = 1. Then we form matrix T of size m x m, such that T resembles the identity matrix as each T[i,i] = 1. Also, k =m, so that when S is assembled all m inmates speak a different language, then \mbox{$\mid$S$\mid$ $\geqslant$ k} is satisfied.\\ 
\textbf{L and k of mute prison problem is satisfiable $\rightarrow$ $\phi$ of 3-SAT is satisfiable}\\
Suppose that T and k of the mute prison problem are satisfiable. Also, suppose $\mid$S$\mid$ is at least m=k. Choose only the first m inmates from S, and extract only their rows from T to form a new matrix T'. It will follows that in T' there will be only m columns where there is at most one entry with the value 1. We will attribute these m columns with variables L$_{1}$, ..., L$_{m}$, such that, 1 $\leqslant$ i $\leqslant$ m, L$_{i}$ = 1. We then form m clauses of a 3-SAT CNF, call them C$_{i}$, ..., C$_{m}$. Each C$_{i}$ relates to L$_{i}$, so that the boolean value of C$_{i}$ = ($z_{i1}$ $\lor$ $z_{i2}$ $\lor$ $z_{i3}$) = 1. Thus set any one of the $z_{i1}$, $z_{i2}$, or $z_{i3}$ to 1. It follows that all C$_{i}$ = 1, thus $\phi$ = (C$_{1}$ $\land$ C$_{2}$ $\land$ ... $\land$ C$_{m}$) Is satisfiable.\\

So, $\phi$ of 3-SAT is satisfiable $\Leftrightarrow$ L and k of mute prison problem is satisfiable . Also, because the reduction was shown to be polynomial it is proven that the mute prison problem is NP-hard. $\blacksquare$\\

By the proofs \underline{1.} and \underline{2.} it follows that the mute prison problem is NP-complete. $\blacksquare$\\

% ----------------------------------------------------------------
% Answer ends
% ----------------------------------------------------------------

\newpage
\item[Q2. The Nonsense Prerequisites]
% ----------------------------------------------------------------
% TODO: Write your answer to the question below. 
% ----------------------------------------------------------------

~\\
\textbf{Claim:} The nonsense prerequisites problem is NP-complete.\\
\textbf{Proof:}\\
\underline{1.} Show the nonsense prerequisites problem is NP.\\
\underline{2.} Show the nonsense prerequisites problem is NP-hard.\\

\underline{1.} Suppose we know G(V, E) and k and we are given E' as a certificate. We verify the certificate with the following algorithm:\\
\begin{algorithm}[H]
 \LinesNumbered 
E'' = E - E'\;
Produce function w, such that $\forall$ (u, v) $\in$ E'', w(u, v) = -1\;
Produce new G'(V, E'', w)\;
\For{v in V}{
Perform Bellman-Ford(G', w, v)\;
\For{each edge (u, v) $\in$ G'.E''}{
\If{v.d \textgreater ~u.d + w(u, v)}{
There is a cycle and the certificate is not satisfiable.
}
}
}
\end{algorithm}
If there is a cycle in G'(V, E'') then setting each edge in G' to a weight -1 will produce a negative edge cycle which, after relaxations, we can identify easily. Given that G(V, E'') may or may not be connected, to locate a cycle in the graph we must perform the relaxation with Bellman-Ford $\mid$V$\mid$ times. Bellman-Ford runs at O(VE), it is executed $\mid$V$\mid$ times in the verifier, thus we have O(V$^2$E) for our algorithm. Since $\mid$V$\mid$ = n, and $\mid$E$\mid$ = O(n$^2$), the verifier runs O(n$^4$). So the verifier is polynomial and then the nonsense prerequisites problem is NP. $\blacksquare$\\ 

\underline{2.} To show the nonsense prerequisites problem is NP-hard, as directed by the problem set, we will perform a reduction using NP-complete problem VECTOR COVER. So, we will show VECTOR COVER $\leqslant_{p}$ The Nonsense Prerequisites Problem.\\

\textbf{Properties of Reduction}\\
Take the G(V, E) and k given to the VECTOR COVER problem. k represents $\mid$S$\mid$ $\leqslant$ k, such that S $\subseteq$ V such that if (u, v) $\in$ E, then u $\in$ S or v $\in$ S. However, in the nonsense prerequisites, the k corresponds to edges that when removed from the graph will make it acyclic. It follows that the reduction must somehow convert the vertices in G to represent edges. This is done by splitting each vertex in two, so given V = \{v$_{1}$, v$_{2}$, ..., v$_{n}$\}, produce V' = \{v$_{pre-1}$, v$_{post-1}$, v$_{pre-2}$, v$_{post-2}$, ..., v$_{pre-n}$, v$_{post-n}$\}, and (v$_{pre-i}$, v$_{post-i}$) is a directed edge such that (v$_{pre-i}$, v$_{post-i}$) $\in$  E'. Also, we must create a circumstance in the new graph where each undirected edge (v$_{i}$, v$_{j}$) $\in$ E, becomes directed edges (v$_{post-i}$, v$_{pre-j}$) $\in$ E' and (v$_{post-j}$, v$_{pre-i}$) $\in$ E'. This construction guarantees in G'(V', E') that when we enter any v$_{pre-i}$ we can walk a path v$_{pre-i}$ $\rightarrow$ v$_{post-i}$ $\rightarrow$ v$_{pre-j}$ $\rightarrow$ v$_{post-j}$ $\rightarrow$ v$_{pre-j}$, and indeed this is a cycle. Thus, we have a cycles, such that if (v$_{i}$, v$_{j}$) $\in$ E, then the cycle is limited to the new vertices \{v$_{pre-i}$, v$_{post-i}$, v$_{pre-j}$, v$_{post-j}$\}. So the reduction is complete and can easily be performed in polynomial time. O(n$\alpha$(m + n)) to produce new directed edges from m existing edges, splitting vertices in V and creating new edges, and adding them to the new graph G' using make-set, union, and link.\\

\textbf{G(V, E), k of VECTOR COVER is satisfiable $\rightarrow$ \\G*(V*, E*), k of the nonsense prerequisite is satisfiable}\\
Suppose using undirected G(V, E) and k, VECTOR COVER is satisfied. Suppose also that we have access to S = \{s$_{1}$, ..., s$_{q}$\}, which is the vertex cover of G and $\mid$S$\mid$ $\leqslant$ k. We perform the reduction and have G*(V*, E*). It follows in G* any cycles is limited to \{v$_{pre-i}$, v$_{post-i}$, v$_{pre-j}$, v$_{post-j}$\}. To break a cycle in G* we could remove any edge from the cycle, but to do this efficiently we need to remove an edges that break many cycles at once. This is precisely E' = \{(s$_{pre-1}$, s$_{post-1}$), ..., (s$_{pre-q}$, s$_{post-q}$)\}, because in G* the edges in E' that correspond to vertices in S, are precisely the set of edges that appear in all cycles. Thus $\mid$E'$\mid$ = $\mid$S$\mid$ $\leqslant$ k, and so G*(V*, E*) and k of the nonsense prerequisite is satisfiable.\\ 
\textbf{G(V, E), k of the nonsense prerequisite is satisfiable $\rightarrow$\\ G*(V*, E*), k of VECTOR COVER is satisfiable}\\
Suppose G(V, E), k when used in the nonsense prerequisite problem is satisfiable. Now, conversively suppose the original graph, G*(V*, E*), before reduction, and k in VERTEX COVER were not satisfiable. This would mean that the set of vertex cover S $\subseteq$ V, $\mid$S$\mid$ textgreater k. But since G(V, E) and k were satisfiable then $\mid$E'$\mid$ $\leqslant$ k. But by the construction of the reduction this in impossible. Since every (v$_{pre-i}$, v$_{post-i}$) $\in$ E' corresponds to a vertex v$_{i}$ $\in$ S, this will mean that there is some v$_{i}$ $\in$ S that is not represented in E', since $\mid$E'$\mid$ \textless ~$\mid$S$\mid$. This means that there is some cycle left over in G when E - E' is performed. So then a contradiction is reached, and so G*(V*, E*), k of VECTOR COVER must be satisfiable.\\

By proving both directions, it follows that G(V, E), k of VECTOR COVER is satisfiable $\leftrightarrow$ \\G*(V*, E*), k of the nonsense prerequisite is satisfiable. $\blacksquare$\\

Additionally, since the reduction can be performed in polynomial time, then the nonsense prerequisite problem is NP-hard. $\blacksquare$


% ----------------------------------------------------------------
% Answer ends
% ----------------------------------------------------------------

\newpage
\item[Q3. T-rex Christmas]
% ----------------------------------------------------------------
% TODO: Write your answer to the question below. 
% ----------------------------------------------------------------

% ----------------------------------------------------------------
% Answer ends
% ----------------------------------------------------------------

\newpage
\item[Q4. Vertex Cover]
% ----------------------------------------------------------------
% TODO: Write your answer to the question below. 
% ----------------------------------------------------------------

% ---------------------------------------------------------------
% Answer ends
% ----------------------------------------------------------------

\end{description}
\end{document}
