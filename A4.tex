% Title:   UofT Art & Sciences Assignment Sample File
% Version: 1.00
% Author:  Kaveh Ghasemloo
% Date:    Sept. 28, 2012
%
% Licence: 
% This work is licensed under the Creative Commons Attribution-ShareAlike 3.0 Unported License. To view a copy of this license, visit http://creativecommons.org/licenses/by-sa/3.0/ or send a letter to Creative Commons, 444 Castro Street, Suite 900, Mountain View, California, 94041, USA.

\documentclass[10pt]{csc_assignment}
\usepackage[]{algorithm2e}
\usepackage{amsmath}
\usepackage{algpseudocode}
\usepackage{qtree}
\usepackage{tkz-graph}

% ----------------------------------------------------------------
% TODO: Enter the assignment number, your name, and your student number below
% ----------------------------------------------------------------
\AssignmentName{4}
\QuestionCount{4}
\StudentName{John Armstrong, Henry Ku}
\StudentNumber{993114492\textbackslash g2jarmst, 998551348\textbackslash g2kuhenr}

% ----------------------------------------------------------------
\begin{document}


\Acknowledgements{
% ----------------------------------------------------------------
% TODO: Write your acknowledgements below.
% ----------------------------------------------------------------

"We declare that we have not used any outside help in completing this assignment."

% ----------------------------------------------------------------
% Aacknowledgements ends
% ----------------------------------------------------------------
}
\begin{description}

\newpage
\item[Q1. The Mute Prison]
% ----------------------------------------------------------------
% TODO: Write your answer to the question below. 
% ----------------------------------------------------------------
~\\
\textbf{Claim:} The mute prison problem is NP-complete.\\
\textbf{Proof:}\\
\underline{1.} Show the mute prison problem is NP.\\
\underline{2.} Show the mute prison problem is NP-hard.\\

\underline{1.} Suppose we are given a certificate S and have access to value k and matrix T. We can verify that the certificate is satisfiable in the following way. Suppose each element in S represents an inmate. Verification would involve iterating on each inmate in the following way:\\
\begin{algorithm}[H]
 \LinesNumbered 
\For {inmate in S}{
j = 1\;
\While{j $\leqslant$ m}{
\If{T[inmate, j]}{
\For {(otherinmate $\neq$ inmate) in S}{
\If{T[otherinmate, j]}{
	S is not a subset of inmates who do don't speak the same language\;
}
}
}
j++\;
}
}
\end{algorithm}
Clearly, the verification that S is a subset where no two inmates speak the same language can run in polynomial time O(mn$^{2}$). Once this verification if complete all that is left to do is to verify that $\mid$S$\mid$ $\geqslant$ k, which is O(1). Therefore the mute prison problem is NP. $\blacksquare$\\

\underline{2.} To show that the mute prison problem is NP-hard we must perform a reduction using an NP-complete problem. We will use a reduction on NP-complete 3-SAT in CNF, in order to show 3-SAT $\leqslant_{p}$ Mute Prison Problem.\\

\textbf{Properties of Reduction}\\
Suppose that $\phi$ is an instance of 3-SAT and C$_{1}$, C$_{2}$, ..., C$_{m}$ are the clauses of $\phi$. By construction of 3-SAT in CNF we have C$_{i}$ = ($z_{i1}$ $\lor$ $z_{i2}$ $\lor$ $z_{i3}$). In the reduction each C$_{i}$'s boolean value will represent a boolean value for each language, L$_{i}$, spoken by some inmate(s), precisely, L$_{i}$ = C$_{i}$ = ($z_{i1}$ $\lor$ $z_{i2}$ $\lor$ $z_{i3}$). Each boolean value for L$_{i}$ has a specific mean:
\[
L_{i} = 
\begin{cases} 
      \hfill 1 \hfill & \text{if L$_{i}$ is spoken by at most 1 inmate} \\
      \hfill 0 \hfill & \text{if L$_{i}$ is spoken by at least 1 inmate} \\
  \end{cases}
\]\\
Producing L$_{1}$, L$_{2}$, ..., L$_{m}$ will take polynomial time since we iterate through each C$_{i}$ and perform a boolean or operation on each z$_{i}$ in C$_{i}$ which takes O(m).\\

Finally, the mute prison problem requires a matrix L to produce the subset of inmates S. Let T be an m x m matrix, so that no inmates are left without a language. The rows in T will represent inmates and the columns will represent languages such that column i represents L$_{i}$. The algortihm that peforms the reduction will iterate through each L$_{i}$. If L$_{i}$ = 1 then set T[i, i] = 1, else if L$_{i}$ =  0 then \mbox{T[1, i] = T[2, i] = ... = T[m, i] = 1}. Assigning all inmates to speak L$_{i}$, when L$_{i}$ = 0, will guarantee that $\mid$S$\mid$ = 0. Alternatively, $\forall$ i, L$_{i}$ = 1 then $\mid$S$\mid$ = m. So that if $\phi$ is satisfies 3-SAT, then T will satisfy the mute prison problem if we set k = m. Again this process is polynomial as it iterates through m L$_{i}$'s and assigns at most m inmates the language L$_{i}$, so it will run O(m$^{2}$). 
% ----------------------------------------------------------------
% Answer ends
% ----------------------------------------------------------------

\newpage
\item[Q2. The Nonsense Prerequisites]
% ----------------------------------------------------------------
% TODO: Write your answer to the question below. 
% ----------------------------------------------------------------

% ----------------------------------------------------------------
% Answer ends
% ----------------------------------------------------------------

\newpage
\item[Q3. T-rex Christmas]
% ----------------------------------------------------------------
% TODO: Write your answer to the question below. 
% ----------------------------------------------------------------

% ----------------------------------------------------------------
% Answer ends
% ----------------------------------------------------------------

\newpage
\item[Q4. Vertex Cover]
% ----------------------------------------------------------------
% TODO: Write your answer to the question below. 
% ----------------------------------------------------------------

% ---------------------------------------------------------------
% Answer ends
% ----------------------------------------------------------------

\end{description}
\end{document}
