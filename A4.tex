% Title:   UofT Art & Sciences Assignment Sample File
% Version: 1.00
% Author:  Kaveh Ghasemloo
% Date:    Sept. 28, 2012
%
% Licence: 
% This work is licensed under the Creative Commons Attribution-ShareAlike 3.0 Unported License. To view a copy of this license, visit http://creativecommons.org/licenses/by-sa/3.0/ or send a letter to Creative Commons, 444 Castro Street, Suite 900, Mountain View, California, 94041, USA.

\documentclass[10pt]{csc_assignment}
\usepackage[]{algorithm2e}
\usepackage{amsmath}
\usepackage{algpseudocode}
\usepackage{qtree}
\usepackage{tkz-graph}

% ----------------------------------------------------------------
% TODO: Enter the assignment number, your name, and your student number below
% ----------------------------------------------------------------
\AssignmentName{4}
\QuestionCount{4}
\StudentName{John Armstrong, Henry Ku}
\StudentNumber{993114492\textbackslash g2jarmst, 998551348\textbackslash g2kuhenr}

% ----------------------------------------------------------------
\begin{document}


\Acknowledgements{
% ----------------------------------------------------------------
% TODO: Write your acknowledgements below.
% ----------------------------------------------------------------

"We declare that we have not used any outside help in completing this assignment."

% ----------------------------------------------------------------
% Aacknowledgements ends
% ----------------------------------------------------------------
}
\begin{description}

\newpage
\item[Q1. The Mute Prison]
% ----------------------------------------------------------------
% TODO: Write your answer to the question below. 
% ----------------------------------------------------------------
~\\
\textbf{Claim:} The mute prison problem is NP-complete.\\
\textbf{Proof:}\\
\underline{1.} Show the mute prison problem is NP.\\
\underline{2.} Show the mute prison problem is NP-hard.\\

\underline{1.} Suppose we are given a certificate S and have access to value k and matrix T. We can verify that the certificate is satisfiable in the following way. Suppose each element in S represents an inmate. Verification would involve iterating on each inmate in the following way:\\
\begin{algorithm}[H]
 \LinesNumbered 
\For {inmate in S}{
j = 1\;
\While{j $\leqslant$ m}{
\If{T[inmate, j]}{
\For {(otherinmate $\neq$ inmate) in S}{
\If{T[otherinmate, j]}{
	S is not a subset of inmates who do don't speak the same language\;
	return 0\;
}
}
}
j++\;
}
}
return 1\;
\end{algorithm}
Clearly, the verification that S is a subset where no two inmates speak the same language can run in polynomial time O(mn$^{2}$). Once this verification is complete all that is left to do is to verify that $\mid$S$\mid$ $\geqslant$ k, which is O(1). Therefore the mute prison problem is NP. $\blacksquare$\\

\underline{2.} To show that the mute prison problem is NP-hard we must perform a reduction using an NP-complete problem. We will use a reduction on NP-complete 3-SAT in CNF, in order to show 3-SAT $\leqslant_{p}$ Mute Prison Problem.\\

\textbf{Properties of Reduction}\\
Suppose that $\phi$ is an instance of 3-SAT and C$_{1}$, C$_{2}$, ..., C$_{m}$ are the clauses of $\phi$. By construction of 3-SAT in CNF we have C$_{i}$ = ($z_{i1}$ $\lor$ $z_{i2}$ $\lor$ $z_{i3}$). In the reduction each C$_{i}$'s boolean value will represent a boolean value for each language, L$_{i}$, spoken by some inmate(s), precisely, L$_{i}$ = C$_{i}$ = ($z_{i1}$ $\lor$ $z_{i2}$ $\lor$ $z_{i3}$). Each boolean value for L$_{i}$ has a specific mean:
\[
L_{i} = 
\begin{cases} 
      \hfill 1 \hfill & \text{if L$_{i}$ is spoken by at most 1 inmate} \\
      \hfill 0 \hfill & \text{if L$_{i}$ is spoken by at least 1 inmate} \\
  \end{cases}
\]\\
Producing L$_{1}$, L$_{2}$, ..., L$_{m}$ will take polynomial time since we iterate through each C$_{i}$ and perform a boolean or operation on each z$_{i}$ in C$_{i}$ which takes O(m).\\

Finally, the mute prison problem requires a matrix T to produce the subset of inmates S. Let T be an m x m matrix, so that no inmates are left without a language. The rows in T will represent inmates and the columns will represent languages such that column i represents L$_{i}$. The algortihm that peforms the reduction will iterate through each L$_{i}$. If L$_{i}$ = 1 then set T[i, i] = 1, else if L$_{i}$ =  0 then set \mbox{T[1, i] = T[2, i] = ... = T[m, i] = 1}. Assigning all inmates to speak L$_{i}$, when L$_{i}$ = 0, will guarantee that $\mid$S$\mid$ = $\emptyset$. Alternatively, $\forall$ i, if L$_{i}$ = 1 then $\mid$S$\mid$ = m. So that if $\phi$ in 3-SAT is satisfiable, then T will satisfy the mute prison problem if we set k = m. Again this process is polynomial as it iterates through m L$_{i}$'s and assigns at most m inmates the language L$_{i}$, so it will run O(m$^{2}$).\\
\textbf{$\phi$ of 3-SAT is satisfiable $\rightarrow$ L and k of mute prison problem is satisfiable}\\
Suppose $\phi$ of 3-SAT is satisfiable, then each clause C$_{1}$, C$_{2}$, ..., C$_{m}$ is satisfied. A set of L$_{1}$, ..., L$_{m}$ is produced such that $\forall$ L$_{i}$, L$_{i}$ = 1. Then we form matrix T of size m x m, such that T resembles the identity matrix as each T[i,i] = 1. Also, k =m, so that when S is assembled all m inmates speak a different language, then \mbox{$\mid$S$\mid$ $\geqslant$ k} is satisfied.\\ 
\textbf{L and k of mute prison problem is satisfiable $\rightarrow$ $\phi$ of 3-SAT is satisfiable}\\
Suppose that T and k of the mute prison problem are satisfiable. Also, suppose $\mid$S$\mid$ = m = k. Suppose T is an m x m matrix that resembles an identity matrix. We will attribute the m columns in T to variables L$_{1}$, ..., L$_{m}$, such that, 1 $\leqslant$ i $\leqslant$ m, and set L$_{i}$ = 1 if the column has at most one entry equal to 1, and set L$_{i}$ = 0 otherwise. Since T and k satisfy the problem then all L$_{i}$ = 1. We then form m clauses of a 3-SAT CNF, call them C$_{i}$, ..., C$_{m}$. Each C$_{i}$ relates to L$_{i}$, so that the boolean value of C$_{i}$ = ($z_{i1}$ $\lor$ $z_{i2}$ $\lor$ $z_{i3}$) = 1. Thus set any one of the $z_{i1}$, $z_{i2}$, or $z_{i3}$ to 1 (or true) to set C$_{i}$ to 1. If a column in T has more than one entry with 1 then clearly the mute prison problem would not be satistified and some C$_{i}$ = 0 (or false) and $\phi$ would not be satisfied. It follows that all C$_{i}$ equal 1 since all L$_{i}$ equal 1, thus $\phi$ = (C$_{1}$ $\land$ C$_{2}$ $\land$ ... $\land$ C$_{m}$) is satisfiable.\\

So, $\phi$ of 3-SAT is satisfiable $\Leftrightarrow$ L and k of mute prison problem is satisfiable . Also, because the reduction was shown to be polynomial it is proven that the mute prison problem is NP-hard. $\blacksquare$\\

By the proofs \underline{1.} and \underline{2.} it follows that the mute prison problem is NP-complete. $\blacksquare$\\

% ----------------------------------------------------------------
% Answer ends
% ----------------------------------------------------------------

\newpage
\item[Q2. The Nonsense Prerequisites]
% ----------------------------------------------------------------
% TODO: Write your answer to the question below. 
% ----------------------------------------------------------------

~\\
\textbf{Claim:} The nonsense prerequisites problem is NP-complete.\\
\textbf{Proof:}\\
\underline{1.} Show the nonsense prerequisites problem is NP.\\
\underline{2.} Show the nonsense prerequisites problem is NP-hard.\\

\underline{1.} Suppose we know G(V, E) and k and we are given E' as a certificate. We verify the certificate with the following algorithm:\\
\begin{algorithm}[H]
 \LinesNumbered 
E'' = E - E'\;
Produce function w, such that $\forall$ (u, v) $\in$ E'', w(u, v) = -1\;
Produce new G'(V, E'', w)\;
\For{v in V}{
Perform Bellman-Ford(G', w, v)\;
\For{each edge (u, v) $\in$ G'.E''}{
\If{v.d \textgreater ~u.d + w(u, v)}{
There is a cycle and the certificate is not satisfiable\;
return 0\;
}
}
}
return 1\;
\end{algorithm}
If there is a cycle in G'(V, E'') then setting each edge in G' to a weight -1 will produce a negative edge cycle which, after relaxations, we can identify easily. Given that G(V, E'') may or may not be connected, to locate a cycle in the graph we must perform the relaxation with Bellman-Ford $\mid$V$\mid$ times. Bellman-Ford runs at O(VE), it is executed $\mid$V$\mid$ times in the verifier, thus we have O(V$^2$E) for our algorithm. Since $\mid$V$\mid$ = n, and $\mid$E$\mid$ = O(n$^2$), the verifier runs O(n$^4$). So the verifier is polynomial and then the nonsense prerequisites problem is NP. $\blacksquare$\\ 

\underline{2.} To show the nonsense prerequisites problem is NP-hard, as directed by the problem set, we will perform a reduction using NP-complete problem VECTOR COVER. So, we will show VECTOR COVER $\leqslant_{p}$ The Nonsense Prerequisites Problem.\\

\textbf{Properties of Reduction}\\
Take the G(V, E) and k given to the VECTOR COVER problem. k represents $\mid$S$\mid$ $\leqslant$ k, such that S $\subseteq$ V such that if (u, v) $\in$ E, then u $\in$ S or v $\in$ S. However, in the nonsense prerequisites, the k corresponds to edges that when removed from the graph will make it acyclic. It follows that the reduction must somehow convert the vertices in G to represent edges. This is done by splitting each vertex in two, so given V = \{v$_{1}$, v$_{2}$, ..., v$_{n}$\}, produce V' = \{v$_{pre-1}$, v$_{post-1}$, v$_{pre-2}$, v$_{post-2}$, ..., v$_{pre-n}$, v$_{post-n}$\}, and $\forall$ i, 1 $\leqslant$ i $\leqslant$ n, (v$_{pre-i}$, v$_{post-i}$) is a directed edge such that (v$_{pre-i}$, v$_{post-i}$) $\in$  E'. Also, we must create a circumstance in the new graph where each undirected edge (v$_{i}$, v$_{j}$) $\in$ E, becomes directed edges (v$_{post-i}$, v$_{pre-j}$) $\in$ E' and (v$_{post-j}$, v$_{pre-i}$) $\in$ E'. This construction guarantees in G'(V', E') that when we enter any v$_{pre-i}$ we can walk a path v$_{pre-i}$ $\rightarrow$ v$_{post-i}$ $\rightarrow$ v$_{pre-j}$ $\rightarrow$ v$_{post-j}$ $\rightarrow$ v$_{pre-j}$, and indeed this is a cycle. Thus, we have a cycles, such that if (v$_{i}$, v$_{j}$) $\in$ E, then the cycle is limited to the new vertices \{v$_{pre-i}$, v$_{post-i}$, v$_{pre-j}$, v$_{post-j}$\}. So the reduction is complete and can easily be performed in polynomial time. O(n$\alpha$(m + n)) to produce new directed edges from m existing edges, splitting vertices in V and creating new edges, and adding them to the new graph G' using make-set, union, and link.\\

\textbf{G(V, E), k of VECTOR COVER is satisfiable $\rightarrow$ \\G*(V*, E*), k of the nonsense prerequisite is satisfiable}\\
Suppose using undirected G(V, E) and k, VECTOR COVER is satisfied. Suppose also that we have access to S = \{s$_{1}$, ..., s$_{q}$\}, which is the vertex cover of G and $\mid$S$\mid$ $\leqslant$ k. We perform the reduction and have G*(V*, E*). It follows in G* any cycles is limited to \{v$_{pre-i}$, v$_{post-i}$, v$_{pre-j}$, v$_{post-j}$\}. To break a cycle in G* we could remove any edge from the cycle, but to do this efficiently we need to remove edges that break many cycles at once. This is precisely E' = \{(s$_{pre-1}$, s$_{post-1}$), ..., (s$_{pre-q}$, s$_{post-q}$)\}, because in G* the edges in E' that correspond to vertices in S, are precisely the set of edges that appear in all cycles. Thus $\mid$E'$\mid$ = $\mid$S$\mid$ $\leqslant$ k, and so G*(V*, E*) and k of the nonsense prerequisite is satisfiable.\\ 
\textbf{G(V, E), k of the nonsense prerequisite is satisfiable $\rightarrow$\\ G*(V*, E*), k of VECTOR COVER is satisfiable}\\
Suppose G(V, E), k when used in the nonsense prerequisite problem is satisfiable. Now, to establish a contradiction, suppose the original graph, G*(V*, E*) and k in VERTEX COVER were not satisfiable. This would mean that the set of vertex cover S $\subseteq$ V, $\mid$S$\mid$ \textgreater k. But since G(V, E) and k were satisfiable then $\mid$E'$\mid$ $\leqslant$ k. But by the construction of the reduction this in impossible. Since every (v$_{pre-i}$, v$_{post-i}$) $\in$ E' corresponds to a vertex v$_{i}$ $\in$ S, this will mean that there is some v$_{i}$ $\in$ S that is not represented in E', since \mbox{$\mid$E'$\mid$ \textless ~$\mid$S$\mid$}. This means that there is some cycle left over in G when E - E' is performed. So then a contradiction is reached based on our original assumption, and so G*(V*, E*), k of VECTOR COVER must be satisfiable.\\

By proving both directions, it follows that G(V, E), k of VECTOR COVER is satisfiable $\leftrightarrow$ \\G*(V*, E*), k of the nonsense prerequisite is satisfiable. $\blacksquare$\\

Additionally, since the reduction can be performed in polynomial time, then the nonsense prerequisite problem is NP-hard. $\blacksquare$

% ----------------------------------------------------------------
% Answer ends
% ----------------------------------------------------------------

\newpage
\item[Q3. T-rex Christmas]
% ----------------------------------------------------------------
% TODO: Write your answer to the question below. 
% ----------------------------------------------------------------
~\\
\underline{1.} \{0, 1\}-integer programming formulation:\\
\underline{Minimize:}\\
\hspace*{1cm}\parbox{15cm}{
P
}\\
\underline{Subject To:}\\
a) \\
\hspace*{1cm}\parbox{15cm}{
x$_{q}^1$ $\in$ \{0, 1\} ~~~~~ $\forall$ q s.t. q represents a path of transferring a present, i $\rightarrow$ j, such that L[i, j] = 1\\
x$_{q}^0$ $\in$ \{0, 1\} 
}\\
b)\\
\hspace*{1cm}\parbox{15cm}{
x$_{q}^1$ + x$_{q}^0$ $\geqslant$ 1\\
}\\
c)\\
\hspace*{1cm}\parbox{15cm}{
$\sum_{\forall q}$ x$_{q}^{f(q,~ m)}$ $\leqslant$ P ~~~~~ $\forall$ m $\in$ \{0, 1, ..., n-1\}\\
}\\
\underline{Explanation of Constraints:}\\
a) x$_{q}^1$ and x$_{q}^0$ represent the direction (x$_{q}^1$ clockwise and x$_{q}^0$ counter-clockwise) a gift q will be passed from i to j, such that i, j $\in$ \{0, 1, ..., n-1\}. If x$_{q}^1$ = 1 then the gift will be sent clockwise, if x$_{q}^1$ = 0 it will not be sent clockwise (same for x$_{q}^0$ but counter-clockwise).\\
b) This constraint limits x$_{q}^1$ and x$_{q}^0$ so both cannot be 0, and it is tightly bound when only one of the variables equals 1.\\
c) Assume some linear function f, and also assume m represents a pass (m, m+1[n]). f(q, m) = 1 if on the clockwise direction of passing a gift from i to j a pass in (m, m+1[n]) occurs. Conversely, f(q, m) = 0 if on the counter-clockwise direction of passing a gift from i to j a pass in (m, m+1[n]) occurs. We can assume linearity of f because the computation of the intersection of gift direction and passes will have been computed previously and values will be contained in some 2-D matrix which f has access to at will. Finally, each summation for each m will represent, when the IP completes, each P$_{m}$. Clearly if the pass (m, m+1[n]) is used during the passing of gift q, then x$_{q}^{f(q,~ m)}$ = 1, the it is added to P$_{m}$. Additionally, the constraint is limited to P, as a tight bound, P is some value that will equal the maximum of all P$_{m}$'s.\\ 

\underline{2.} LP relaxation:\\
\underline{Minimize:}\\
\hspace*{1cm}\parbox{15cm}{
P
}\\
\underline{Subject To:}\\
a) \\
\hspace*{1cm}\parbox{15cm}{
0 $\leqslant$ x$_{q}^1$ $\leqslant$ 1 ~~~~~ $\forall$ q s.t. q represents a path of transferring a present, i $\rightarrow$ j, such that L[i, j] = 1\\
0 $\leqslant$ x$_{q}^0$ $\leqslant$ 1 
}\\
b)\\
\hspace*{1cm}\parbox{15cm}{
x$_{q}^1$ + x$_{q}^0$ $\geqslant$ 1\\
}\\
c)\\
\hspace*{1cm}\parbox{15cm}{
$\sum_{\forall q}$ x$_{q}^{f(q,~ m)}$ $\leqslant$ P ~~~~~ $\forall$ m $\in$ \{0, 1, ..., n-1\}\\
}\\
\underline{Rounding Scheme:}\\
Let all x$_{1}^1$*, x$_{1}^0$*, x$_{2}^1$*, x$_{2}^0$*, ..., x$_{s}^1$*, x$_{s}^0$* (s being the number of gifts to be exchanged) and P be the solution returned by the relaxed LP. Allow us to round all x$_{q}^1$*'s and x$_{q}^0$*'s to produce y$_{1}^1$, y$_{1}^0$, y$_{2}^1$, y$_{2}^0$, ..., y$_{s}^1$, y$_{s}^0$, and P'. So, let this represent the solution to the integer programming such that:\\
\[
y_{q}^1 ~or~ y_{q}^0 = 
\begin{cases} 
      \hfill 1 \hfill & \text{if corresponding x$_{q}^1$* or x$_{q}^0$* $\geqslant$ $\frac{1}{2}$} \\
      \hfill 0 \hfill & \text{otherwise} \\
  \end{cases}
\]
\center{P' = max$_{0 \leqslant r \leqslant n - 1}$ (P'$_{r}$ = $\sum_{\forall q}$ y$_{q}^{f(q,~ r)}$)} \\
\begin{flushleft}
\underline{Proof that Rounding Scheme works:}\\
All constraints in the interger program will be satisfied. For example a) is satisfied because y$_{q}^1$ or y$_{q}^0$ $\in$ \{0, 1\}. Constraint b) is satisfied even in the unfortunate case where x$_{q}^1$* and x$_{q}^0$* equals 1/2 or greater, then y$_{q}^1$ and y$_{q}^0$ will equal 1. Additionally both y$_{q}^1$ and y$_{q}^0$ cannot both equal 0 by the constraint b). Finally, in constraint c), P is a bound specified by the algorithm, and indeed with the specified values for y$_{q}^1$ and y$_{q}^0$ the constraint will be satisfied by replacing P with P'. Therefore we conclude that the result of our rounding scheme will be feasible in the integer program.\\
\underline{Proof of 2-approximation:}\\
In both the relaxed linear program and rounding scheme function f will return the same results, such that the result of the relaxed linear program, since P is minimized, P will equal some P$_{m}$, for passes at (m, m+1[n]). Indeed suppose \mbox{P = P$_{m}$ = $\sum_{\forall q}$ x$_{q}^{f(q,~ m)}$}*. It follows, from our rounding scheme, that the rounding solution must be P' = \mbox{P'$_{r}$ = $\sum_{\forall q}$ y$_{q}^{f(q,~ r)}$} representing the number of passes at some (r, r+1[n]). Then the proof is such that:\\
x$_{q}^1$* $\geqslant$ $\frac{1}{2}$ $\rightarrow$ y$_{q}^1$ = 1\\
x$_{q}^0$* $\geqslant$ $\frac{1}{2}$ $\rightarrow$ y$_{q}^0$ = 1\\
then,\\
2x$_{q}^1$* $\geqslant$ 1 $\rightarrow$ 2x$_{q}^1$* $\geqslant$ y$_{q}^1$\\
2x$_{q}^0$* $\geqslant$ 1 $\rightarrow$ 2x$_{q}^0$* $\geqslant$ y$_{q}^0$\\
finally,\\
P'$_{r}$ = $\sum_{\forall q}$ y$_{q}^{f(q,~ r)}$ $\leqslant$
2*P$_{r}$ = 2*$\sum_{\forall q}$ x$_{q}^{f(q,~ r)}$\\ 
P'$_{r}$ = $\sum_{\forall q}$ y$_{q}^{f(q,~ r)}$
$\leqslant$
2*P$_{m}$ = 2*$\sum_{\forall q}$ x$_{q}^{f(q,~ m)}$~~~~~~~~\# Since P$_{r}$ $\leqslant$ P$_{m}$\\
P'$_{r}$ = $\sum_{\forall q}$ y$_{q}^{f(q,~ r)}$ $\leqslant$ 2*P $\leqslant$  2*LPOPT $\leqslant$ 2*IPOPT ~~~~~~~~\# Since we established that P = P$_{m}$\\
Therefore, the rounding scheme attains a 2-approximation. $\blacksquare$\\
 
% ----------------------------------------------------------------
% Answer ends
% ----------------------------------------------------------------

\newpage
\item[Q4. Vertex Cover]
% ----------------------------------------------------------------
% TODO: Write your answer to the question below. 
% ----------------------------------------------------------------
~\\
\underline{1.} \{0, 1\}-integer programming formulation:\\
\underline{Minimize:}\\
\hspace*{1cm}\parbox{15cm}{
$\sum_{u \in V}$ C$_{v}$(u)x$_{u}$ + $\sum_{(u, v) \in E}$ C$_{e}$(u,v)x$_{(u, v)}$
}\\
\underline{Subject To:}\\
a) \\
\hspace*{1cm}\parbox{15cm}{
x$_{u}$ $\in$ \{0, 1\} ~~~~~ $\forall$ u $\in$ V\\ 
x$_{(u, v)}$ $\in$ \{0, 1\} ~~~~~ $\forall$ (u, v) $\in$ E\\ 
}\\
b)\\
\hspace*{1cm}\parbox{15cm}{
x$_{u}$ + x$_{v}$ + x$_{(u, v)}$ $\geqslant$ 1\\
}\\
\underline{Explanation of Constraints:}\\

a)\\
 x$_{u}$ $\in$ \{0, 1\} represents whether or not vertex u is  in our vertex cover S.\\
 If x$_{u}$ = 0 then it is not in our vertex cover, and if x$_{u}$ = 1, then it is in our vertex cover.\\
~\\
The second constraint x$_{(u, v)}$ $\in$ \{0, 1\} represents whether or not the edge (u, v) is covered by the vertices in our vertex cover S. If x$_{(u, v)}$ = 0 then it is not covered, if x$_{(u, v)}$ = 1 then it is covered by our vertex cover.\\
~\\
b)\\
 x$_{u}$ + x$_{v}$ + x$_{(u, v)}$ $\geqslant$ 1 shows that x$_{u}$, x$_{v}$ , x$_{(u, v)}$ that one of them must at least be covered or in our vertex cover S. This constraints make sure that if the edge is (u,v) is not in our S, then either x$_{u}$ or x$_{v}$, one of them must have the value 1 and viceversa.\\
~\\

\underline{2.} LP relaxation:\\
\underline{Minimize:}\\
\hspace*{1cm}\parbox{15cm}{
$\sum_{u \in V}$ C$_{v}$(u)x$_{u}$ + $\sum_{(u, v) \in E}$ C$_{e}$(u,v)x$_{(u, v)}$
}\\
\underline{Subject To:}\\
a) \\
\hspace*{1cm}\parbox{15cm}{
0 $\leqslant$ x$_{u}$ $\leqslant$ 1 ~~~~~ $\forall$ u $\in$ V\\ 
0 $\leqslant$ x$_{(u, v)}$ $\leqslant$ 1 ~~~~~ $\forall$ (u, v) $\in$ E\\ 
}\\
b)\\
\hspace*{1cm}\parbox{15cm}{
x$_{u}$ + x$_{v}$ + x$_{(u, v)}$ $\geqslant$ 1\\
}\\
\underline{Rounding Scheme:}\\
$\forall$ u $\in$ V, and $\forall$ (u, v) $\in$ E let all x$_{u}*$ and x$_{(u, v)}*$ be the solution to the relaxed LP. Allow us to round all x$_{u}*$'s and x$_{(u, v)}*$'s to the interger programming solution, such that all y$_{u}$'s and y$_{(u, v)}$'s will be a feasible IP solution:\\
\[
y_{u} = 
\begin{cases} 
      \hfill 1 \hfill & \text{if corresponding x$_{u}$* $\geqslant$ $\frac{1}{3}$} \\
      \hfill 0 \hfill & \text{otherwise} \\
  \end{cases}
\]\\
\[
y_{(u, v)} = 
\begin{cases} 
      \hfill 1 \hfill & \text{if corresponding x$_{(u, v)}$* $\geqslant$ $\frac{1}{3}$} \\
      \hfill 0 \hfill & \text{otherwise} \\
  \end{cases}
\]\\
\underline{Proof that Rounding Scheme works:}\\
\textbf{HENRY TO DO}\\
\underline{Proof of 3-approximation:}\\
\textbf{HENRY TO DO (add intro text if needed)}\\
x$_{u}$* $\geqslant$ $\frac{1}{3}$ $\rightarrow$ y$_{u}$ = 1\\
x$_{(u, v)}$* $\geqslant$ $\frac{1}{3}$ $\rightarrow$ y$_{(u, v)}$ = 1\\
then,\\
3x$_{u}$* $\geqslant$ 1 $\rightarrow$ 3x$_{u}$* $\geqslant$ y$_{u}$\\
3x$_{(u, v)}$* $\geqslant$ 1 $\rightarrow$ 3x$_{(u, v)}$* $\geqslant$ y$_{(u, v)}$\\
finally,\\
$\sum_{u \in V}$ C$_{v}$(u)y$_{u}$ + $\sum_{(u, v) \in E}$ C$_{e}$(u,v)y$_{(u, v)}$ $\leqslant$
$\sum_{u \in V}$ C$_{v}$(u)(3x$_{u}$) + $\sum_{(u, v) \in E}$ C$_{e}$(u,v)(3x$_{(u, v)}$)\\
$\sum_{u \in V}$ C$_{v}$(u)y$_{u}$ + $\sum_{(u, v) \in E}$ C$_{e}$(u,v)y$_{(u, v)}$ $\leqslant$
3($\sum_{u \in V}$ C$_{v}$(u)x$_{u}$) + 3($\sum_{(u, v) \in E}$ C$_{e}$(u,v)x$_{(u, v)}$)\\
$\sum_{u \in V}$ C$_{v}$(u)y$_{u}$ + $\sum_{(u, v) \in E}$ C$_{e}$(u,v)y$_{(u, v)}$ $\leqslant$
3($\sum_{u \in V}$ C$_{v}$(u)x$_{u}$ + $\sum_{(u, v) \in E}$ C$_{e}$(u,v)x$_{(u, v)}$)\\
$\sum_{u \in V}$ C$_{v}$(u)y$_{u}$ + $\sum_{(u, v) \in E}$ C$_{e}$(u,v)y$_{(u, v)}$ $\leqslant$ 3*LPOPT $\leqslant$ 3*IPOPT\\
Therefore, the rounding scheme attains a 3-approximation. $\blacksquare$\\
% ---------------------------------------------------------------
% Answer ends
% ----------------------------------------------------------------
\end{flushleft}
\end{description}
\end{document}
